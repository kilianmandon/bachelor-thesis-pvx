\subsection{DNA Cloning}
DNA cloning was concluded by sequencing of the plasmids and the calculation of the yield from the midi preparation. The full agreement of the sequencing result with the designed sequences enables further use of the plasmids in cell-free expression. The yields of the midi preparation are slightly lower than the manufacturer's reported typical yield of \SI{500}{\micro\gram}, but still high enough to allow for use in cell-free expression after concentration. 

\subsection{Protein Expression}
\subsubsection{eYFP Yield by Fluorescence}
Following cell-free expression, the protein yield of the batch was estimated by fluorescence measurements of the Strep-eYFP reaction setup. The determined concentration of \SI{0.3}{\milli\gram\per\milli\litre} is significantly lower than the manufacturer's reported typical yield of at least \SI{2}{\milli\gram\per\litre}. Typical reasons for low lysate performance, as stated by the manufacturer, are an inappropriate amount of plasmid DNA, poor oxygenation, or expired lysate. As for the plasmid DNA, the manufacturer explains that the optimal amount varies for each protein and should be determined by testing. Potentially, use of less DNA could improve the yield. Oxygenation in this setup was provided through a hole in the reaction tube's lid. An optimal oxygen supply could be established by using the dedicated perforated lids shipped with newer versions of the lysate. Use of newer lysate might also lead to a higher yield, since deterioration of the kit's components is a known issue 

\subsubsection{Coomassie Staining and Western Blot}
Evaluation of the cell-free expression's qualitative success, as well as of the CaptoCore chromatography, was conducted through Western Blots using an anti-PVX antibody and an anti-S-Tag antibody.

After the Coomassie staining of the ALiCE lysate, the samples d29-A, S-Tag-A, and S-Tag-B, showed no notable difference from the non-template control, while the tracks containing lysate from the PVX-CP and the Strep-eYFP setup display bands at molecular weights matching the control. This could imply a lower yield in these setups. However, the coloring from other proteins in the lysate provides bad contrast, and visual differences might simply be caused by the fact that the bands are at slightly different heights than those of Strep-eYFP, and possibly worse to distinguish from the background. Further, the control S-Tag-CP-PVX only displays a faint band as well, reinforcing that there is no sure implication on the yield. 

The anti-PVX Western Blot shows defined bands for d29-A, S-Tag-A, and PVX-CP, but only a very faint band for S-Tag-B. This could be due to a lower yield for S-Tag-B, or less avidity of the anti-PVX antibody against the protein, given that in only has \SI{0.53}{\percent} sequence identity to the wild type. For all samples and the control S-Tag-CP-PVX, the migration of distance suggested a molecular weight higher than the theoretical one. For PVX particles from plants, this is a well-known phenomenon and partly due to specific glycosylations of the proteins \cite{pvx_glycosylation}. However, glycosylation of the ALiCE samples is unlikely, since they were expressed through the cytosolic pathway. The discrepancy with the theoretical weights might imply partial aggregation.  

In the Western Blot using the anti-S-Tag antibody, both samples S-Tag-A and S-Tag-B show a strong signal, implying that the antibody has high avidity against them. The signal is not constrained to the migration distance as seen in the anti-PVX blot, but is faintly visible through most of the track. Signal at larger molecular weights might stem from aggregates, while signal at lower molecular weights suggests the existence of smaller protein parts containing an S-Tag, possibly through partial degradation of the protein while boiling the samples. 

Following CaptoCore purification, no discernible bands were visible in the Coomassie gel. However, the Western Blot showed bands for S-Tag-A and PVX-CP using the anti-PVX antibody, as well as a faint band for S-Tag-A and a very faint band for S-Tag-B when using the anti-S-antibody. This implies the presence of particles or protein aggregates with a molecular weight larger than \SI{700}{\kilo\Dalton} in the samples, even though in low amounts. Notably, the blot also shows a band for PVX-CP, even though the wild type coat protein is known to not assemble into VLPs in absence of a suitable RNA \cite{pvx_assembly}. The faint bands could thus stem either from assembled particles, or from protein aggregates with a high molecular weight. This was further analyzed by electron microscopy.


\subsubsection{ELISA}
For a quantitative analysis of the cell-free expression of d29-A, S-Tag-A, S-Tag-B, and PVX-CP, ELISAs using both anti-PVX and anti-S-Tag antibodies were conducted. 

In both cases, the optical densities of the samples were almost identical for the 1:500 and the 1:1000 dilution. This might be caused by other proteins in the lysate saturating the binding capacity of the microwell plate, even at the stronger dilution, and thus rendering only the proportion of the specific protein in the lysate relevant. Under this assumption, the result from the 1:1000 dilution would be more significant, even though in general stronger dilutions should be tested to get a better estimate. 

Using the 1:1000 dilution, the highest protein concentration in the anti-PVX ELISA was observed for the PVX-CP, with a calculated concentration of \SI{134}{\micro\gram\per\milli\litre}. This is less than half the value measured for Strep-eYFP expression by the lysate. A possible cause is lower expression or quicker degradation of the protein in the lysate. A different reason could be that the antibody has higher avidity for the assembled viral particles than for the PVX-CP in the ALiCE lysate, which are known to not assemble without specific RNA present \cite{pvx_assembly}. This is plausible, since the antibody was generated by rabbit immunization against viral particles. For the constructs d29-A, S-Tag-A, and S-Tag-B, the anti-PVX ELISA leads to lower calculated concentrations of about \SI{30}{\micro\gram\per\milli\litre} for d29-A and S-Tag-A, and no discernible signal for S-Tag-B. This could also be due to lower avidity of the antibody towards the adapted sequences. The missing signal for S-Tag-B is consistent with the anti-PVX Western Blot, which showed only a very faint line for S-Tag-B. 

Using the anti-S-Tag antibody, only the samples from S-Tag-A and S-Tag-B had significant optical densities. The back-calculated concentrations of \SI{2.4}{\milli\gram\per\milli\litre} and \SI{1.2}{\milli\gram\per\milli\litre} are very high, and unlikely given the measured expression of only \SI{0.3}{\milli\gram\per\milli\litre} for the Strep-eYFP ALiCE control setup. Possibly, the anti-S-Tag antibody has considerably higher avidity against protein monomers than for the assembled particles in the calibration samples. This could explain the high optical density, since the faint signal in the CaptoCore after the Western Blot implies that little to no protein in the ALiCE setup assembled. Notably, the anti-S-Tag antibody is not explicitly labeled as being suitable for ELISA, so the high signal might also be due to this, even though the calibration showed a linear trend with high determination.

\subsubsection{Electron Microscopy}

