RFdiffusion is a generative machine learning model for protein backbone design. It can be run in different modes to accomplish several tasks such as unconditional monomer generation, protein binder design, scaffolding around a fixed motif, or design of symmetric oligomers (Figure \ref{fig:rfdiff_trajectories}), the latter being the most relevant for this work. In practice, RFdiffusion is commonly used together with ProteinMPNN, where RFdiffusion generates synthetic backbone structure and ProteinMPNN tries to realize this backbone with a synthetic amino acid sequence. 

\begin{figure}
\centering
\includesvg[width=\textwidth]{modeling/rfdiffusion_traj_basic.svg}
\caption{\textbf{RFdiffusion trajectories for unconditional generation and symmetric noise.} For unconditional generation, RFdiffusion samples the initial positions independently from a Gaussian distribution and generates the protein without any constraints. For the generation of oligomers with a specific symmetry, RFdiffusion only samples coordinates for one monomer and initializes the other coordinates by applying the respective symmetry transform to the coordinates of that reference monomer. In each diffusion step, this is repeated to enforce the symmetry. }
\label{fig:rfdiff_trajectories}
\end{figure}

Generation by RFdiffusion is performed through a reverse Riemannian diffusion process on the manifold $\mathrm{SE}(3)$. Compared to other diffusion-based algorithms like AlphaFold3 (Section \ref{ch:alphafold}), RFdiffusion doesn't operate on the atom coordinates using standard euclidean diffusion, but diffuses the backbone transforms instead. However, it converts the transforms from and to atom coordinates in each iteration. For unconditional generation, the model starts with randomly initialized backbone coordinates and creates a based solely on a specified number of residues. For the creation of symmetric oligomers, the user specifies a set of transforms $\mathfrak{R}=\{R_k\}_{k=1}^K \in \mathrm{SO}(3)$ that define the symmetry. The final protein will satisfy $x^{(k)} = R_k x^{(1)}$, where $x^{(k)}$ denotes the coordinates of the $k$-th monomer. This is achieved by explicitly setting the coordinates as such after initialization and in each further iteration (algorithm \ref{alg:rfdiff}). 

\begin{algorithm}
    \caption{Generation of symmetric oligomers}
    \begin{algorithmic}[1]
    \AlgFunctionDef{SampleSymmetric}{$M, \mathfrak{R} = \{R_k\}_{k=1}^K$}
    \AlgComment{RFdiffusion generation of oligomer with symmetry $\mathfrak{R}$}
    \State $x^{(T,1)} = \text{SampleReference}(M)$
    \ForAll{$t = T, \ldots, 1$}
        \AlgComment[1]{Symmetrize chains}
        \State $X^{(t)} = [R_1 x^{(t,1)}, \ldots, R_K x^{(t,1)}]$
        \State $\hat{X}^{(0)} = \text{RFdiffusion}(X^{(t)})$
        \State $[x^{(t-1,1)}, \ldots, x^{(t-1,K)}] = \text{ReverseStep}(X^{(t)}, \hat{X}^{(0)})$
    \EndFor
    \State \Return $\hat{X}^{(0)}$
    \end{algorithmic}
    \label{alg:rfdiff}
\end{algorithm}

In the original RFdiffusion code, only point group symmetries are supported, that is symmetries that satisfy $\{x^{(1)}, ..., x^{(K)}\} = \{R_j x^{(1)}, ..., R_j x^{(K)}\}$ for each $R_j$, up to reordering of the monomers. Technically, the code could work on general euclidean transforms $T_j \in \mathrm{SE}(3)$ (such as the transforms from Section \ref{ch:symmetry} specifying the symmetry of PVX) as well. However, there are certain drawbacks in doing so. The positions in early steps in the diffusion process follow a Gaussian distribution. While the rotations by point group symmetries conserve that distribution, general euclidean transforms don't. This means that the positions that are fed into the noise prediction network in the diffusion process don't follow the distribution the model is trained on. Further, the authors of RFdiffusion observed that for point group symmetries, the noise prediction model conserves the symmetry almost perfectly. Due to this, the explicit symmetrization in each iteration is generally not necessary and barely affects the trajectory, if the initial noise is symmetrized. This arises from the equivariance of the $\mathrm{SE}(3)$-transformer architecture used in RFdiffusion. Using the symmetry transforms for PVX as evaluated in Section \ref{ch:symmetry}, the atom positions in early stages of the diffusion process don't follow a Gaussian distribution, and same-seeded trajectories using either full symmetry enforcement or only initially symmetrized noise differ strongly (Figure \ref{fig:rfdiff_sym_trajectories}). Rather, for initial symmetrization, the atoms quickly collapse to a Gaussian-like distribution, before spreading out again to form the final multimer.

\begin{figure}[!hbtp]
    \centering
    \includesvg[width=\textwidth]{modeling/rfdiffusion_traj_symmetry.svg}
    \caption{\textbf{RFdiffusion trajectories for full and initial symmetry enforcement. }}
    \label{fig:rfdiff_sym_trajectories}
\end{figure}

\FloatBarrier

RFdiffusion with full symmetry enforcement of the PVX symmetry was used to generate backbone structures for further testing. In total, five different backbone structures where designed, and ProteinMPNN was used as described in Section \ref{ch:pmpnn} to generate five sequences for each of them. Additionally, backbone structures were generated using partial denoising, where only a limited amount of noise was added to the wild type structure before denoising again. This was done using 5, 10, 15, and 20 noise steps in RFdiffusion. For each of these noise levels, three denoised backbones were generated using RFdiffusion, each realized by three sequences through ProteinMPNN. The de novo generated backbones typically consist of a simple structure of alpha helices and beta sheets (Figure \ref{fig:rfdiffusion_examples}). The sequences were further evaluated using the methods described in sections \ref{ch:alphafold} and \ref{ch:gromacs}.


\begin{figure}
\centering
\includesvg[width=\textwidth]{modeling/rfdiffusion_examples.svg}
\caption{\textbf{RFdiffusion examples. }}
\label{fig:rfdiffusion_examples}
\end{figure}

Possibly due to the aforementioned incongruities in running the algorithm with euclidean transforms, backbone structures generated for the PVX symmetry tend to have structural violations, in particular interchain clashes. 
